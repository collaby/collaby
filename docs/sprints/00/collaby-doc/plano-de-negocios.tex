\section{Introdução}

Num mundo cada vez mais \emph{online} pessoas necessitam cada vez mais
ferramentas onde possam trabalhar juntas sem a necessidade de estarem no
mesmo lugar. Com essa ideia surgiram ferramentas colaborativas como
Google Docs, que permite editar Documentos, Planilhas e Apresentações
com vários usuários envolvidos. Daí surge outra necessidade, uma
ferramenta que se encaixe no âmbito científico. Surge então o Projeto
Collaby.

\subsection{Finalidade}

\begin{itemize}
\item
  Proporcionar um local central de informações compartilhadas
\item
  Descentralizar a resolução de exercícios
\item
  Criar um local especializado na geração de documentos científicos
\item
  Incentivar alunos a utilizarem LaTeX desde cedo
\item
  Incentivar uso de tecnologias como markdown na criação de documentos
\item
  Aumentar o nível dos projetos haja vista que o número de projetos
  diferentes deverá aumentar.
\end{itemize}
\subsection{Definições, Acrônimos e Abreviações}

\begin{itemize}
\item
  Front-end: é a parte do sistema de software que interage diretamente
  com o usuário.
\item
  \emph{free}: termo que, em software, significa liberdades em relação a
  distribuição do código-fonte, criado por Richard M. Stallman.
\item
  \emph{open-source}: propriedade intelectual que é disponível
  livremente através de licença pública por seus criadores.
\item
  ajax: Asynchronous JavaScript and XML.
\end{itemize}
\subsection{Referências}

\begin{itemize}
\item
  \href{https://togetherjs.com/}{TogetherJS: torna página colaborativa}
\item
  \href{http://daringfireball.net/projects/markdown/dingus}{Definição da
  linguagem de marcação Markdown}
\item
  \href{http://johnmacfarlane.net/pandoc/}{Pandoc: converte documentos
  em Markdown para PDF, HTML}
\item
  \href{http://www.mathjax.org/}{Mathjax: renderiza fórmulas matemáticas
  LaTeX no browser}
\item
  \href{http://framework.zend.com/zf2}{Zend Framework 2 - ambiente web
  estável e completo}
\item
  \href{http://getbootstrap.com/}{Twitter Bootstrap 3 - front-end}
\item
  \href{http://jquery.com/}{jQuery 2 - eventos, ajax}
\item
  \href{http://ace.c9.io/\#nav=about}{Ace Editor}
\item
  \href{http://FontAwesome.github.io/}{FontAwesome 4 - icon web fonts}
\item
  \href{https://github.com/chjj/marked}{Marked}
\end{itemize}
\subsection{Visão Geral}

\section{Descrição do Produto}

Collaby é uma ferramenta web para aprendizado colaborativo, onde é
possível compartilhar exercícios, listas, materiais de estudo,
apresentação de slides, etc.

\section{Contexto do Negócio}

Funcionará no contexto acadêmico, tanto para discentes quanto docentes.
É um projeto \emph{free} e \emph{open-source}.

\paragraph{Mas porque usar linguagem Markdown?}

Essa resposta pode ser obtida vendo o uso do Markdown na internet. Veja
alguns sites que a utilizam:

\begin{itemize}
\item
  \href{http://github.com}{Github} - README e outros documentos; Wiki;
\item
  \href{http://stackoverflow.com/}{Stackoverflow} - perguntas e
  respostas;
\item
  \href{https://leanpub.com/authors\#how\_leanpub\_works}{Leanpub} -
  autoria de livros;
\item
  \href{https://ghost.org/}{Ghost} - blogging platform;
\item
  \href{https://github.com/joemccann/dillinger}{dillinger} - editor
  \emph{online} com suporte a upload para Dropbox, Google Drive ou
  Github.
\item
  entre outros.
\end{itemize}
Além disso hoje temos a necessidade de ter livros, artigos ou outros
tipos de texto disponíveis para Desktop, Tablet e smartphones. Dessa
forma usar Markdown centraliza o foco no texto e deixa a formatação para
outras ferramentas como o pandoc.

\section{Objetivos do Produto}

\begin{itemize}
\item
  Exercícios, materiais de aula, etc. são criados e compartilhados.
\item
  Documentos podem ser editados em equipe utilizando TogetherJS
\item
  Documentos são públicos para leitura, podendo também ser públicos para
  escrita se assim o usuário desejar.
\item
  Documentos são escritos em markdown com suporte a fórmulas matemáticas
  em LaTeX usando pandoc, MathJax.
\item
  Documentos podem ser exportados para PDF ou HTML usando pandoc.
\item
  Lista de links de referência.
\item
  Importar/Exportar de um arquivo markdown.
\item
  Documentos tem suporte a linguagens de programação, podendo o código
  ser embutido e mostrado com o devido highlighting.
\item
  Documentos podem ser apresentações em LaTeX beamer ou slides em HTML5.
\end{itemize}
\section{Estimativas Financeiras}

Gastos incluem:

\begin{itemize}
\item
  Domínio
\item
  Hospedagem
\end{itemize}
\section{Restrições}

\begin{itemize}
\item
  A colaboratividade pode ser afetada caso a ferramenta TogetherJS seja
  descontinuada.
\item
  O editor web codemirror caso deixe de ser mantido pode deixar de
  funcionar nos browsers cada vez mais modernos.
\end{itemize}
