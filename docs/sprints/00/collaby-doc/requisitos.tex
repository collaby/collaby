\section{Requisitos}

\subsection{Requisitos Funcionais}

\subsubsection{Página inicial \label{home}}

Mapeada como:
\texttt{/ {[}Application\textbackslash{}Controller\textbackslash{}Index.index{]}}

\paragraph{Usuário deslogado}

Deve mostrar uma lista com os últimos documentos com a seguinte ordem:
última atualização descendente.

Na lista deve aparecer o nome do documento, um link para visualização,
nome do dono do documento, link para o usuário dono, data de
atualização, tags para o documento, link para buscar documento por tag.

Caso o documento seja um clone mostrar o link para documento original.

\paragraph{Usuário logado}

Deve mostrar os documentos do usuário logado, permitindo link de edição
ao invés de visualização.

\subsubsection{Login}

Mapeada como:
\texttt{/login {[}Application\textbackslash{}Controller\textbackslash{}Auth.login{]}}

Página de login tradicional usando usuário e senha, com adição da opção
de login com a conta do \emph{Twitter}, \emph{Facebook} ou
\emph{Google+}.

Links úteis:

\begin{itemize}
\item
  \url{https://dev.twitter.com/docs/auth/sign-twitter}
\item
  \url{https://developers.facebook.com/docs/facebook-login/login-flow-for-web/}
\item
  \url{https://developers.google.com/+/quickstart/javascript}
\end{itemize}
Colocar opções para Redefinir Senha \ref{reset-pass} e Criar Conta
\ref{signup}.

\subsubsection{Redefinir Senha \label{reset-pass}}

Mapeado como:
\texttt{/reset-password {[}Application\textbackslash{}Controller\textbackslash{}Account.reset-password{]}}

Caso o usuário esqueça a senha é necessário que ele a redefina. Para
isso ele deve preencher uma tela com os dados:

\begin{itemize}
\item
  nome de usuário ou e-mail
\item
  Captcha, para garantir que não é um robô.
\end{itemize}
Em seguida um \emph{hash} de recuperação de senha é criado com a
validade de 1 dia. Daí um e-mail é enviado para o usuário com os dados
da solicitação: mensagem, link de recuperação, validade do link.

Ao acessar o link, verifica se o \emph{hash} é válido e mostra a tela
para digitar nova senha.

\subsubsection{Criar Conta \label{signup}}

Mapeado como:
\texttt{/signup {[}Application\textbackslash{}Controller\textbackslash{}Account.signup{]}}

Criar uma conta deve ser muito simples. A página requisita apenas:

\begin{itemize}
\item
  nome de usuário
\item
  senha
\item
  e-mail
\end{itemize}
Em seguida um e-mail será enviado para o e-mail informado contendo uma
mensagem de boas vindas e o link para a validação do cadastro. O link é
composto pela url do site seguida de um \emph{hash}. Ao fazer a
requisição o \emph{hash} é verificado com no banco de dados e o usuário
é validado.

O \emph{hash} deve ser criado a partir do nome do usuário, o e-mail e um
\emph{salt}, que é um pequeno \emph{hash}.

Referências sobre \emph{salt}:

\begin{itemize}
\item
  \href{https://crackstation.net/hashing-security.htm\#salt}{Salted
  Password Hashing - Doing it Right}
\item
  \href{http://en.wikipedia.org/wiki/Salt\_(cryptography)}{Salt
  (cryptography)}
\end{itemize}
O nome de usuário deve ser único, assim como o e-mail. Assim sendo antes
de criar a conta deve-se verificá-los antes de prosseguir com a criação
da conta.

\subsubsection{Novo Documento}

Mapeado como:
\texttt{/new {[}Application\textbackslash{}Controller\textbackslash{}Document.new{]}}

Quando o usuário requisitar um novo documento, o mesmo é criado com
valores padrão e logo em seguida o usuário é levado a Editar Documento
\label{edit-document}.

Ao editar o documento pela primeira vez, uma tela modal aparece para que
o usuário possa editar o nome do documento e escolher um tema. Ao final
ele clica em um botão ``Ok''.

Daí ele vai para a edição do documento em si \ref{edit-document}.

\subsubsection{Editar Documento \label{edit-document}}

Mapeado como:
\texttt{/d/:id {[}Application\textbackslash{}Controller\textbackslash{}Document.edit{]}}

\paragraph{Usuário deslogado ou usuário logado que não é dono do
documento}

Redireciona usuário para modo de visualização (\emph{read-only})
\ref{view-document}.

\paragraph{Usuário logado e dono do documento}

Abre o documento em modo de edição.

Esta tela deve ser aberta em nova aba, isso se deve ao fato de que ela
será uma tela diferenciada e além disso será uma tela que pode ser
compartilhada para que outras pessoas possam editá-la ao mesmo tempo de
forma colaborativa. Além disso faz com que não seja possível usar o
botão ``Voltar'' do browser, tendo em vista que o histórico não existe
numa aba aberta pela primeira vez.

Ela possui um editor de texto com suporte a \emph{highlight} para
\textbf{Markdown}.

Deve haver um menu com as opções de Salvar e Pré-visualizar.

Do lado esquerdo um menu para definir qual arquivo editar,
\textbf{content} ou \textbf{template}.

Logo abaixo a estrutura do arquivo em forma de árvore, onde aparecerão
links para as linhas que iniciam com \#.

E no rodapé uma caixa de mensagens, com o \emph{log} para detectar
possíveis erros ao gerar o ducumento em pdf.

\subsubsection{Exportar Documento}

Mapeado como:
\texttt{/d/:id/export{[}/:type{]} {[}Application\textbackslash{}Controller\textbackslash{}Document.export{]}}

Opção vista em Editar Documento \ref{edit-document} e Visualizar
Documento \ref{view-document}.

Permite exportar o documento para PDF ou HTML, sendo o tipo padrão é
PDF. Ao gerar o arquivo ele deve ser mandando para o browser em forma de
download.

Caso o arquivo seja HTML, ele deve ser mandado como um arquivo único.
Isso significa que se ele possui CSS ou Javascript deve vir embutido no
arquivo.

\subsubsection{Importar Documento}

Mapeado como:
\texttt{/import {[}Application\textbackslash{}Controller\textbackslash{}Document.import{]}}

Página destinada a usuários logados que desejam submeter um arquivo
Markdown existente e a partir dele criar um documento.

Além de um formulário contendo um \emph{input file} deve permitir o nome
do arquivo bem como o layout a ser usado.

\subsubsection{Clonar Documento}

Mapeado como:
\texttt{/clone/:id {[}Application\textbackslash{}Controller\textbackslash{}Document.clone{]}}

Todo documento deve ser público, sendo posível assim clonar qualquer um
deles. É possível clonar um documento a partir da página Visualizar
Documento \ref{view-document}.

É um processo simples onde todas as informações do documento são
copiadas de um usuário para outro. Todavia, como um clone, a edição dos
arquivos se torna independente.

\subsubsection{Visualizar Usuário}

Mapeado como:
\texttt{/u/:username {[}Application\textbackslash{}Controller\textbackslash{}User.view{]}}

Página destinada a mostrar dados e os documentos de um usuário. De onde
outros usuário podem ver e clonar seus documentos.

Os documentos tem a mesma visualização da página inicial \ref{home}.

\subsubsection{Visualizar Documento \label{view-document}}

Mapeado como:
\texttt{/view/:id {[}Application\textbackslash{}Controller\textbackslash{}Document.view{]}}

Visualização HTML do documento \textbf{Markdown} com suporte a fórmulas
matemáticas (MathJax).

\subsection{Requisitos Não-funcionais}

\subsubsection{Traduzir a ferramenta TogetherJS}

Por ser uma ferramenta muito nova ainda não possui tradução nem mesmo
está sujeita ainda. Devido a isso temos duas opções:

\begin{itemize}
\item
  Traduzir os arquivos de forma bruta (Mais fácil mas de difícil
  manutenção pois quando a ferramenta for atualizada temos que sair de
  arquivo em arquivo retraduzindo).
\item
  Criar um mecanismo de tradução que irá ajudar a ferramenta para que
  seja possível traduzir para qualquer língua (Mais difícil mas podemos
  ser os primeiros a tornar a ferramenta traduzível e contribuir com um
  programa da Mozilla).
\end{itemize}
\subsubsection{Sincronização de cursor do Ace Editor}

Existe um problema com a ferramenta TogetherJS em conjunto com o Ace
Editor no qual quando o documento é atualizado o cursor dos outros
participantes é resetado para o início do documento.

Outros detalhes em:

\begin{itemize}
\item
  \url{https://github.com/mozilla/togetherjs/pull/927}
\item
  \url{https://github.com/triglian/togetherjs}
\item
  \href{http://ace.c9.io/\#nav=about}{Ace Editor}
\end{itemize}
