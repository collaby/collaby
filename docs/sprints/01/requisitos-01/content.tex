\section{Requisitos}

Estes são os requisitos escolhidos para serem feitos no sprint 01.

\subsection{Requisitos Funcionais}

\subsubsection{Login}

Mapeada como:
\texttt{/login {[}Application\textbackslash{}Controller\textbackslash{}Auth.login{]}}

Responsável: Átila

Página de login tradicional usando usuário e senha.

Colocar opções para Criar Conta \ref{signup}.

\subsubsection{Criar Conta \label{signup}}

Mapeado como:
\texttt{/signup {[}Application\textbackslash{}Controller\textbackslash{}Account.signup{]}}

Responsável: Péricles

Criar uma conta deve ser muito simples. A página requisita apenas:

\begin{itemize}
\item
  nome de usuário
\item
  senha
\item
  e-mail
\end{itemize}
Em seguida um e-mail será enviado para o e-mail informado contendo uma
mensagem de boas vindas e o link para a validação do cadastro. O link é
composto pela url do site seguida de um \emph{hash}. Ao fazer a
requisição o \emph{hash} é verificado com no banco de dados e o usuário
é validado.

O \emph{hash} deve ser criado a partir do nome do usuário, o e-mail e um
\emph{salt}, que é um pequeno \emph{hash}.

Referências sobre \emph{salt}:

\begin{itemize}
\item
  \href{https://crackstation.net/hashing-security.htm\#salt}{Salted
  Password Hashing - Doing it Right}
\item
  \href{http://en.wikipedia.org/wiki/Salt\_(cryptography)}{Salt
  (cryptography)}
\end{itemize}
O nome de usuário deve ser único, assim como o e-mail. Assim sendo antes
de criar a conta deve-se verificá-los antes de prosseguir com a criação
da conta.

\subsubsection{Novo Documento}

Mapeado como:
\texttt{/new {[}Application\textbackslash{}Controller\textbackslash{}Document.new{]}}

Responsável: Átila

Quando o usuário requisitar um novo documento, o mesmo é criado com
valores padrão e logo em seguida o usuário é levado a Editar Documento
\label{edit-document}.

Ao editar o documento pela primeira vez, uma tela modal aparece para que
o usuário possa editar o nome do documento e escolher um tema. Ao final
ele clica em um botão ``Ok''.

Daí ele vai para a edição do documento em si \ref{edit-document}.

\subsubsection{Editar Documento \label{edit-document}}

Mapeado como:
\texttt{/d/:id {[}Application\textbackslash{}Controller\textbackslash{}Document.edit{]}}

Responsável: Jovane

\paragraph{Usuário logado e dono do documento}

Abre o documento em modo de edição.

Esta tela deve ser aberta em nova aba, isso se deve ao fato de que ela
será uma tela diferenciada e além disso será uma tela que pode ser
compartilhada para que outras pessoas possam editá-la ao mesmo tempo de
forma colaborativa. Além disso faz com que não seja possível usar o
botão ``Voltar'' do browser, tendo em vista que o histórico não existe
numa aba aberta pela primeira vez.

Ela possui um editor de texto com suporte a \emph{highlight} para
\textbf{Markdown}.

Deve haver um menu com as opções de Salvar e Pré-visualizar. Além de
opções de formatação como negrito, itálico, links, código \emph{inline},
desfazer, refazer.
